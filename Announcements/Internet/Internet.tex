%============================== Setting teh Document=============================================

\documentclass[aspectratio=169]{beamer}
\usepackage[italian]{babel} 
\usepackage[utf8]{inputenc} 
\usepackage[T1]{fontenc}
\usepackage{graphicx}
\usepackage{xcolor}
\usetheme{Copenhagen}
\usepackage{multicol}
\usepackage[pscoord]{eso-pic}
\usepackage{setspace}
\usepackage{tikz}

%==========================Set the foot line========================================================
\makeatletter

\setbeamercolor*{author in head/foot}{parent=palette tertiary}
\setbeamercolor*{title in head/foot}{parent=palette primary}
\setbeamercolor*{date in head/foot}{parent=palette primary}

\setbeamercolor*{section in head/foot}{parent=palette tertiary}
\setbeamercolor*{subsection in head/foot}{parent=palette primary}
% colors for the external link field
\setbeamercolor*{LinkToIndex}{parent=palette tertiary}

\defbeamertemplate*{footline}{}
{
	\leavevmode%
	\hbox{%
		\begin{beamercolorbox}[wd=.25\paperwidth,ht=2.25ex,dp=1ex,center]{author in head/foot}%
			\usebeamerfont{author in head/foot}%\insertsection
		\end{beamercolorbox}%
	
		\begin{beamercolorbox}[wd=.25\paperwidth,ht=2.25ex,dp=1ex,center]{title in head/foot}\centering{-Tel-}\hspace*{2ex}
			%\insertsubsection
		\end{beamercolorbox}%
	
	% this is a new field with an external link
	    \begin{beamercolorbox}[wd=.25\paperwidth,ht=2.25ex,dp=1ex,center]{LinkToIndex}%
	    	\usebeamerfont{author in head/foot}\centering{\href{mailto:f.rombaldoni@campus.uniurb.it}{f.rombaldoni@campus.uniurb.it}}\hspace*{2ex}
    	\end{beamercolorbox}%

		\begin{beamercolorbox}[wd=.25\paperwidth,ht=2.25ex,dp=1ex,center]{date in head/foot}%
			\usebeamerfont{date in head/foot}\centering{Francesco Rombaldoni}\hspace*{2em}
		\end{beamercolorbox}}%	

	\vskip0pt%
}
\setbeamersize{text margin left=1em,text margin right=1em}
\makeatother

\AddToShipoutPictureFG{
	\put(\LenToUnit{.884\paperwidth},
	\LenToUnit{.25\paperheight})
	{\vtop{{\null}
			\makebox{\begin{tikzpicture}
					% << Replace with your personal image >>
					\clip (0,0) circle (6mm) node {\includegraphics[width=22mm]{Imgs/temp}};
\end{tikzpicture}}}}} 

\setbeamercovered{dynamic}

\title{Offerta Di Ripetizioni} 
\author{Francesco Rombaldoni} 

%===========================================Document starting===============================================
\begin{document}

\section{Potenziamento Rete Internet}
\setbeamertemplate{navigation symbols}{}
\begin{frame}[t]{\textbf{Potenziamento della rete Internet senza cambiare operatore!}}
	
	\mbox{
	\colorbox{gray!20}{\begin{minipage}[t][0.8\textheight][t]
			{\dimexpr0.4\textwidth-2\fboxsep-2\fboxrule-5pt\relax}
			\centering{\textbf{\textcolor{red}{Pensi che il tuo internet non sia veloce quanto ti avevamo promesso?} \newline \textcolor{blue}{Allora stai guardando la locandina giusta!}}
				\newline
				\scriptsize{Sono un informatico appassionato di tecnologia, ed ho da poco assemblato uno strumento che si integra facilmente nella rete locale di casa permettendo di potenziarla, se anche tu pensi che la connessione si potrebbe migliorare,} \textbf{Non esitare a contattarmi!}
			}
	\end{minipage}}
		\colorbox{gray!20}{\begin{minipage}[t][0.8\textheight][t]
			{\dimexpr0.4\textwidth-2\fboxsep-2\fboxrule-5pt\relax}
		\textbf{Servizi offerti dal sispositivo una volta installato}
		{\scriptsize
		\begin{itemize}
			\item Cancellazione della pubblicità su domini noti
			\item Ottimizzazione della rete internet 
			\item Anonimizzazione rispetto la navigazione dal fornitore di Internet
		\end{itemize}
		}
		\textbf{Tipo di supporti}
		{\scriptsize
		\begin{itemize}
			\item Se hai un modem \textbf{libero}, basta installarlo.
			\item Altrimenti, installo una rete ottimizzata (LAN e WiFi 6) a cascata dal modem.
		\end{itemize}}
	\vspace*{-3mm}
		\begin{figure}
			\centering
			\includegraphics[width=0.25\textwidth]{Imgs/internet.png}
		\end{figure}
	\end{minipage}}


\colorbox{gray!20}{\begin{minipage}[t][0.8\textheight][t]
		{\dimexpr0.22\textwidth-2\fboxsep-2\fboxrule-5pt\relax}
		\centering{\textbf{Contattami su WhatsApp}}
			\begin{figure}
			\centering
			\includegraphics[width=0.5\textwidth]{Imgs/WhatsApp-QR-code}
		\end{figure}
		\centering{\textbf{Contattami su Telegram}}
	\begin{figure}
		\centering
		\includegraphics[width=0.5\textwidth]{Imgs/WhatsApp-QR-code}
	\end{figure}
	\end{minipage}}
}


\end{frame}

\end{document}
